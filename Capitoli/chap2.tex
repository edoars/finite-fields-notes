%!TEX root = ../main.tex
%%%%%%%%%%%%%%%%%%%%%%%%%%%%%%%%%%%%%%%%%
%
%LEZIONE 09/03/2018 - TERZA SETTIMANA (2)
%
%%%%%%%%%%%%%%%%%%%%%%%%%%%%%%%%%%%%%%%%%
\chapter{Polynomials over Finite Fields}

\section{Order of polynomial and primitive polynomials}

\begin{lem}\label{3.1}
	Let \(f\in \F_q[x]\) be a polynomial of degree \(m\ge 1\) with \(f(0)\neq 0\). Then there exists \(e\in \N^+, e\le q^m-1\) such that
	\[
		f(x) \mid x^e-1.
	\]
\end{lem}

\begin{proof}
	Consider the residue class ring
	\[
		R = \frac{\F_q[x]}{(f)} = \Set{a_0+a_1\a + \ldots + a_{m-1}\a^{m-1} | a_i \in \F_q, \a \text{ root of }f}.
	\]
	\(R\) has \(q^m-1\) nonzero elements. Now consider the \(q^m\) residue classes
	\[
		x^j + (f) \qquad\text{with }0 \le j \le q^m-1,
	\]
	which are all nonzero because \(f(0)\neq 0\). In particular there exists \(r,s \in \N^+, 0 \le r < s \le q^m-1\) such that
	\[
		x^r + (f) = x^s+(f) \iff x^r \equiv x^s \pmod{f},
	\]
	hence \(f\) divides \(x^s-x^r = x^r(x^{s-r}-1)\). Moreover \(\GCD(x,f)=1\) as \(f(0)\neq 0\), and so
	\[
		f \mid x^r(x^{s-r}-1) \implies f \mid x^{s-r}-1.
	\]
	Now define \(e=s-r\) and \(f\) divides \(x^e-1\) with \(0<e\le q^m-1\).
\end{proof}

\begin{defn}{Order of polynomial}{3.2}\index{Order!of polynomial}
	Let \(f(x) \in \F_q[x]\) with \(f \not\equiv 0\). If \(f(0)\neq 0\), we define the \emph{order} of \(f\) as the least positive integer \(e\) such that \(f\) divides \(x^e-1\):
	\[
		\ord(f) = \min\Set{i \in \N^+ | f(x) \mid x^i-1}.
	\]
	If \(f(0)=0\), write \(f(x)=x^h g(x)\) with \(h\in\N^+\) and \(g(x)\in\F_q[x]\) such that \(g(0)\neq 0\). Then define the order of \(f\) as the order of \(g\).
\end{defn}

\begin{ese}
	Let \(f(x)=x^k,k\ge 0, f \in \F_q[x]\). In this case
	\[
		f(x) = x^k g(x) \qquad\text{with }g(x)=1.
	\]
	Therefore the order of \(f\) is \(\ord(f)=\ord(g)=1\).
\end{ese}

\begin{ese}
	Let \(f(x)=x^2+x+1 \in \F_2[x]\). It is necessary to compute \(\ord(f)\) by hand. Observe that \(\ord(f)\ge \pd f =2\). Clearly \(f\) does not divide \(x^2+1\), but is easy to show that \(f(x) \mid x^3+1\) (As \(f=Q_3\) and \(x^3+1=Q_1 Q_3\)). Therefore \(\ord(f)=3\).
\end{ese}

\begin{teor}{Order of polynomial equal to the order of its roots}{3.3}
	Let \(f\in \F_q[x]\) be an irreducible polynomial of degree \(m\) with \(f(0)\neq 0\) and let \(\a\) be any root of \(f\). Then the order of \(f\) is equal to the order of \(\a\) in \(\F_{q^m}^*\).
\end{teor}

\begin{proof}
	As \(f\) is an irreducible polynomial of degree \(m\), \(\F_{q^m}\) is the splitting field of \(f\) over \(\F_q\). By \autoref{th:2.18}, any root of \(f\) has the same order in \(\F_{q^m}^*\). Let \(\a\) be any root of \(f\), from \autoref{2.12} we know that
	\[
		\a^e = 1 \iff f(x) \mid x^e-1.
	\]
	The claim follows if we take \(e\) the least positive integer with this property.
\end{proof}

\begin{cor}\label{3.4}
	Let \(f\in \F_q[x]\) be an irreducible polynomial of degree \(m\). Then
	\[
		\ord(f) \mid q^m-1.
	\]
\end{cor}

\begin{proof}
	If \(f(0)\neq 0\), then, by previous theorem,
	\[
		\ord(f) = \ord_{\F_{q^m}^*}(\a) \mid q^m-1,
	\]
	as \(\F_{q^m}^*\) is a group of order \(q^m-1\).
	If \(f(0)=0\), then \(f\) irreducible implies
	\[
		f(x) = c\,x \qquad\text{with }c \in \F_q.
	\]
	Therefore \(\ord(f)=1 \mid q-1\).
\end{proof}

\begin{ese}
	Let \(f(x)=x^3-x^2+1 \in \F_3[x]\) which is irreducible as it does not have roots in \(\F_3\). By previous theorem, we can find the order of \(f\) computing the order of one of its roots \(\a\) in \(\F_{3^3}^*\). Now
	\[
		\ord(\a) \mid 3^3-1 = 26 \implies \ord(\a) \in \{1,2,13,26\}.
	\]
	Moreover \(\ord(\a) \ge \pd f = 3\), hence \(\ord(\a) \in \{13,26\}\).
	Then it is enough to compute \(\a^{13}=\a^8\a^4\a\), with \(\a^3=\a^2-1\). Now
	\[
		\a^4 = \a\,(\a^2-1) = \a^3-\a = \a^2-\a-1 = \a^2+2\a+2
	\]
	And
	\begin{align*}
		\a^8 & = (\a^4)^2 = (\a^2+2\a+2)^2 = \a^4+\a^2+1+\a^3+\a^2+2\a \\
		     & = \a^4+\a^3+2\a^2+2\a+1 = \a^2+2\a+2+\a^2+2+2\a^2+2\a+1 \\
		     & = \a^2+\a+2
	\end{align*}
	Therefore
	\begin{align*}
		\a^13 & = \a^8\a^4\a = (\a^2+\a+2)(\a^2+2\a+2)\a = \a\,(\a^4+1) \\
		      & = \a\,(\a^2+2\a+2+1) = \a\,(\a^2+2\a) = \a^3+2\a^2      \\
		      & = \a^2-1+2\a = -1.
	\end{align*}
	Hence \(\ord(f)=\ord(\a)=26\).
\end{ese}

\begin{teor}{}{3.5}
	Let \(A_{m,e}\) be the set of polynomials in \(\F_q[x]\) which are monic, irreducible, with degree \(m\) and order \(e\). Then
	\[
		\abs{A_{m,e}} =
		\begin{cases}
			\frac{\j(e)}{m} & \text{if }e \ge 2 \text{ and }m = \ord_{\Z_e}(q) \\
			2               & \text{if }e = m = 1                              \\
			0               & \text{otherwise}
		\end{cases}
	\]
\end{teor}

\begin{proof}
	Let \(f\in \F_q[x]\) be a monic irreducible polynomial of degree \(m\). If \(\a\) is a root of \(f\), by previous theorem we know that
	\[
		\ord(f) = \ord_{\F_{q^m}^*}(\a) = e \iff \a^e=1.
	\]
	This is equivalent to saying that all roots of \(f\) are primitive \(e\)-th root of unity over \(\F_q\). In particular \(f\) must divide \(Q_e\). But from \autoref{th:2.47} we also know that each monic irreducible factor of \(Q_e\) has as a degree the least positive integer such that \(q^s \equiv 1\) modulo \(e\), hence \(m=\ord_{\Z_e}(q)\). From the same theorem we also know that there are \(\j(e)/m\) of such factors.
	
	If \(m=e=1\) the only possibilities for \(f\) are given by
	\[
		f(x) =x-1 \qquad\text{and}\qquad f(x)=x.
	\]
	Therefore \(\abs{A_{1,1}=2}\).
\end{proof}

\begin{lem}\label{3.6}
	Let \(c \in \N^+\) and \(f \in \F_q[x]\) with \(f(0)\neq 0\). Then
	\[
		f(x) \mid x^c-1 \iff \ord(f) \mid c.
	\]
\end{lem}

\begin{proof}
	\graffito{\("\Leftarrow"\)}Let \(e=\ord(f)\) and suppose \(e \mid c\). Then 
	\[
		e = \ord(f) \iff f(x) \mid x^e-1 \qquad\text{and}\qquad e \mid c \iff x^e-1 \mid x^c-1,
	\]
	therefore \(f\) divides \(x^c-1\).
	
	\graffito{\("\Rightarrow"\)}Suppose that \(f\) divides \(x^c-1\), then \(c \ge e\). We can write
	\[
		c = m\,e + r \qquad\text{with }m,r\in \N^+ \text{ and }0\le r <e.
	\]
	Then
	\[
		x^c-1 = x^{m\,e+r}-1 = x^{m\,e+r}-1+x^r-x^r = x^r (x^{m\,e}-1) + (x^r-1).
	\]
	Now \(f\) divides \(x^e-1\), hence it divides \(x^{m\,e}-1\), therefore
	\[
		f(x)\mid x^c-1, x^{m\,e}-1 \implies f(x) \mid x^r-1.
	\]
	But \(r<e\), so \(r=0\) by definition of order. Hence \(e \mid c\).
\end{proof}

\begin{cor}
	Let \(e_1,e_2 \in \N^+\). Then, in \(\F_q[x]\),
	\[
		\GCD(x^{e_1}-1,x^{e_2}-1) = x^d-1,
	\]
	with \(d = \GCD(e_1,e_2)\).
\end{cor}

\begin{proof}
	Let \(f\) be the \(\GCD(x^{e_1}-1,x^{e_2}-1)\). Now \(d=\GCD(e_1,e_2)\) implies
	\[
		x^d-1 \mid x^{e_1}-1\qquad\text{and}\qquad x^d-1 \mid x^{e_2}-1,
	\]
	hence \(x^d-1\) divides \(f(x)\). On the other hand, as \(f\) divides \(x^{e_1}-1\) and \(x^{e_2}-1\), from previous lemma we have
	\[
		\ord(f) \mid e_1 \qquad\text{and}\qquad \ord(f) \mid e_2.
	\]
	Therefore \(\ord(f)\) divides \(\GCD(e_1,e_2)=d\) and so \(f\) divides \(x^d-1\).
\end{proof}

\begin{teor}{Order of powers of a polynomial}{3.8}
	Let \(g\in\F_q[x]\) be an irreducible polynomial of order \(e\) with \(g(0)\neq 0\) and let \(f = g^b\) with \(b\in \N^+\). Then \(f\) has order \(p^t e\), where \(p\) is the characteristic of \(\F_q\) and
	\[
		t = \min\Set{i \in \N^+ | p^i \ge b}.
	\]
\end{teor}

\begin{proof}
	Let \(c\) be the order of \(f\), so that \(f\) divides \(x^c-1\). Then
	\[
		g(x) \mid \big(g(x)\big)^b = f(x) \mid x^c-1 \iff e \mid c,
	\]
	by \autoref{3.6}. Now \(g\) divides \(x^e-1\) so \(g^b\) divides \((x^e-1)^b\); by definition of \(t\)
	\[
		p^t \ge b \implies (x^e-1)^b \mid (x^e-1)^{p^t}.
	\]
	But \(F_q\) has characteristic \(p\), therefore
	\[
		(x^e-1)^{p^t} = x^{e\,p^t}-1 \implies f(x) = \big(g(x)\big)^b \mid x^{e\,p^t}-1,
	\]
	hence \(c \mid e\,p^t\).
	Now observe that \(e \mid c\) so we can write \(c = k\,e\), then
	\[
		c \mid e\,p^t \iff k\,e \mid e\,p^t \implies k\mid p^t,
	\]
	so \(k = p^j\) with \(0 \le j \le t\) and \(c = e\,p^j\).
	Note that, by \autoref{3.4}, \(e\) divides \(q^m-1\), with \(m\) the degree of \(g\), therefore \(e\) does not divide \(p\) and \(x^e-1\) has only simple roots. Therefore
	\[
		x^c-1 = x^{e\,p^j}-1 = (x^e-1)^{p^j}
	\]
	has \(e\) distinct roots, each of them with multiplicity \(p^j\).
	But every root of \(f=g^b\) has multiplicity \(b\) and
	\[
		f(x) \mid (x^e-1)^{p^j} \implies b \le p^j.
	\]
	However, by construction, the least positive \(j\) for this to happen is \(t\). But we have already seen that \(j \le t\), so
	\[
		j = t \qquad\text{and}\qquad c=p^t e.\qedhere
	\]
\end{proof}

\begin{teor}{Computing the order of a polynomial}{3.9}
	Let \(g_1,\ldots,g_k\in\F_q[x]\) be pairwise relatively prime nonzero polynomial and let \(f=g_1 \cdot\ldots\cdot g_k\). Then
	\[
		\ord(f) = \lcm\big(\ord(g_1),\ldots,\ord(g_k)\big).
	\]
\end{teor}
%%%%%%%%%%%%%%%%%%%%%%%%%%%%%%%%%%%%%%%%%%
%
%LEZIONE 13/03/2018 - QUARTA SETTIMANA (1)
%
%%%%%%%%%%%%%%%%%%%%%%%%%%%%%%%%%%%%%%%%%%
\begin{proof}
	Let \(e_i=\ord(g_i)\) and \(e=\lcm(e_1,\ldots,e_k)\). By \autoref{3.6}
	\[
		g_i(x) \mid x^{e_i}-1 \mid x^e-1 \qquad\text{for all }i.
	\]
	Therefore \(\lcm(g_1,\ldots,g_k)=f \mid x^e-1\). Now let \(c = \ord(f)\), then \(c \mid e\). As \(g_i\) are factors of \(f\), we have
	\[
		f(x) \mid x^c-1 \implies g_i(x) \mid x^c-1 \implies e_i \mid c \qquad\text{for all }i.
	\]
	Therefore \(e\mid c\).
\end{proof}

\begin{ese}
	Consider the following polynomial in \(\F_2[x]\):
	\[
		f(x) = (x^2+x+1)^3 (x^4+x+1) = g(x)^3 h(x).
	\]
	We know by previous examples that \(g\) is primitive, therefore \(g\) has order \(\ord(\a)=3\) with \(\a\) a root of \(g\). \(h\) is also primitive and has order \(15\) as its roots.
	The order of \(g^3\) is \(\ord(g)p^t\), with \(t\) the least positive integer such that \(p^t\ge 3\). Therefore \(\ord(g^3)=\ord(g)2^2 = 12\). By the previous theorem we have
	\[
		\ord(f) = \lcm(12,15) = 60.
	\]
\end{ese}

\begin{cor}
	Let \(\F_q\) be a finite field with characteristic \(p\) and let \(f \in \F_q[x]\) with \(f(0)\neq 0\).
	Suppose \(f = a\,f_1^{b_1} \cdot\ldots\cdot f_k^{b_k}\), where \(a \in \F_q\) and \(f_i \in \F_q[x]\) irreducible and distinct polynomials with \(b_i \ge i\) for all \(i\). Then
	\[
		\ord(f) = \lcm\big(\ord(f_1), \ldots, \ord(f_k)\big) p^t,
	\]
	with \(t\) the least positive integer such that \(p^t \ge \max\{b_1,\ldots,b_k\}\).
\end{cor}

\begin{oss}
	In general, factorize \(f\) could be difficult, so we want another method of determining the order of \(f\).
	Recall that the order of \(f\) is defined as the least positive integer \(e\) such that \(f\) divides \(x^e-1\). Hence, in general, we can reduce \(x^i\) modulo \(f\) or compute the order of \(x\) in \(\F_q[x]/(f)\) (which is not always a field).
	
	Now assume that \(f\) is irreducible with degree \(m\) and order \(e\). By \autoref{3.4} we know that \(e\) divides \(q^m-1\), which can be easily factored even for big values of \(q\) and \(m\). Say
	\[
		q^m-1 = p_1^{r_i} \cdot\ldots\cdot p_s^{r_s},
	\]
	then we can check if
	\[
		x^{\frac{q^m-1}{p_i}} \not\equiv 1 \pmod{f}.
	\]
	In this case \(e\) is a multiple of \(p_i^{r_i}\). If instead it reduces to \(1\) modulo \(f\), then \(e\) is not a multiple of \(p_i^{r_i}\) and we can check whether \(e\) is a multiple of \(p_i^{r_i-1}, p_i^{r_i-2}, \ldots, p_i\), by calculating the residues modulo \(f\) of
	\[
		x^{\frac{q^m-1}{p_i^2}}, x^{\frac{q^m-2}{p_i^3}}, \ldots, x^{\frac{q^m-1}{p_i^{r_i}}}.
	\]
	We can repeat this computation for each prime factor of \(q^m-1\) to obtain the factorization of \(e\).
\end{oss}

\begin{defn}{Reciprocal polynomial}{3.12}\index{Reciprocal polynomial}
	Let \(f(x) = a_0 + a_1 x + \ldots + a_{n-1}x^{n-1} + a_n x^n\) be a polynomial in \(\F_q[x]\). The \emph{reciprocal polynomial} \(f^*\) of \(f\) is defined as
	\[
		f^*(x) = x^n f\left(\frac{1}{x}\right) = a_0 x^n + a_1 x^{n-1} + \ldots + a_{n-1}x + a_n.
	\]
\end{defn}

\begin{oss}
	If \(f(0) \neq 0\), then \(\a \in V(f)\) if and only if \(1/\a \in V(f^*)\). Conversely, if \(f(0)=0\), write \(f(x) = x^h g(x)\) with \(g(0)\neq 0\), then
	\[
		f^*(x) = x^n \frac{1}{x^h}g\left(\frac{1}{x}\right) = x^{n-h}g\left(\frac{1}{x}\right) = g^*(x).
	\]
\end{oss}

\begin{teor}{Order of the reciprocal polynomial}{3.13}
	Let \(f \in \F_q[x]\) be a nonzero polynomial and \(f^*\) be its reciprocal polynomial. Then
	\[
		\ord(f) = \ord(f^*).
	\]
\end{teor}

\begin{proof}
	Suppose \(f(0)\neq 0\) and let \(e=\ord(f)\). If \(\a\) is a root of \(f\), then \(a^e = 1\) and also \((1/\a)^e=1\), where \(1/\a\) is a root of \(f^*\), therefore
	\[
		f \mid x^e -1 \implies f^* \mid x^e-1.
	\]
	In the same way we can prove that if \(f^*\) divides \(x^e-1\) then also \(f\) does.
	If \(f(0)=0\), write \(f(x)=x^h g(x)\), then by definition of order and from the previous observation, we have
	\[
		\ord(f) = \ord(g) = \ord(g^*) = \ord(f^*). \qedhere
	\]
\end{proof}

\begin{notz}
	Let \(f\) be a polynomial in \(\F_q[x]\). We say that \(f\) is \emph{even} if all irreducible factors of \(f\) have even order. Otherwise we say that \(f\) is odd.
\end{notz}

\begin{teor}{Order of \(f(-x)\)}{3.14}
	Consider \(\F_q\) with \(q\) odd, let \(f \in \F_q[x]\) be a polynomial with \(f(0)\neq 0\) and let \(F(x) = f(-x)\). Let \(e=\ord(f)\) and \(E=\ord(F)\), then
	\[
		\begin{cases}
			E=e     & e \equiv 0 \pmod{4}                                 \\
			E = 2e  & e \equiv 1 \pmod{4} \text{ or } e \equiv 3 \pmod{4} \\
			E = e/2 & e \equiv 2 \pmod{4} \text{ and \(f\) even}          \\
			E = e   & e \equiv 2 \pmod{2} \text{ and \(f\) odd}
		\end{cases}
	\]
\end{teor}

\begin{proof}
	Since \(\ord(f)=e\), then by \autoref{3.6}, \(f\) divides \(x^{2e}-1\), hence
	\[
		F \mid (-x)^{2e}-1 = x^{2e}-1 \implies E \mid 2e.
	\]
	But we can easily invert the role of \(f\) and \(F\) to obtain that \(e\) divides \(2E\). Therefore
	\[
		E/e \in \Set{1,2,1/2}.
	\]
	\begin{itemize}
		\item Suppose \(e \equiv 0 \pmod{4}\), then \(e\) is even, therefore
		      \[
			      f \mid x^e-1, F\mid (-x)^e-1 = x^e-1 \implies E \mid e.
		      \]
		      Moreover \(E\) is even, as \(e=4k\) and \(E/e \in \{1,2,1/2\}\). Therefore
		      \[
			      F \mid x^E-1, f \mid (-x)^E-1 = x^E-1 \implies e \mid E,
		      \]
		      hence \(E=e\).
		\item Suppose \(e \equiv 1,3 \pmod{4}\), then
		      \[
			      f \mid x^e-1, F \mid (-x)^e-1 = -(x^e+1).
		      \]
		      Clearly \(F\) can not divide also \(x^e-1\), otherwise
		      \[
			      F \mid \GCD(x^e-1,x^e+1) =1.
		      \]
		      Hence \(E \nmid e\), and knowing \(E/e \in \{1,2,1/2\}\) implies \(E=2e\).
		\item Suppose \(e \equiv 2 \pmod{4}\), hence \(e = 2h\) with \(h\) odd. Consider \(f=g^b\) with \(g\) an irreducible polynomial in \(\F_q[x]\). Note that
		      \[
			      f \mid x^{2h}-1 = (x^h-1)(x^h+1),
		      \]
		      so \(g\) divides either \(x^h-1\) or \(x^h+1\), but not both as they do not have common factors.
		      Now if \(g\mid x^h-1\), then \(g^b \mid x^h-1\) which is impossible as \(f\) has order \(2h\). Therefore
		      \[
			      g \mid x^h+1 \implies g^b=f \mid x^h+1 \implies F \mid (-x)^h +1 = -(x^h-1),
		      \]
		      hence \(E=e/2\). Note that we are necessarily in the case of \(f\) even as, by \autoref{th:3.8}, the power of an irreducible polynomial has even order if and only if the irreducible polynomial itself has even order (and \(\Char(\F_q)\neq 2\)).
		      
		      In general we have \(f = g_1 \cdot\ldots\cdot g_k\) with \(g_i\) is a power of an irreducible polynomial and \(g_1,\ldots,g_k\) are pairwise relatively prime. By \autoref{th:3.9}
		      \[
			      \ord(f) = 2h = \lcm\big(\ord(g_1),\ldots,\ord(g_k)\big).
		      \]
		      We reorganize \(g_1,\ldots,g_k\) in such a way that \(g_i\) has even order \(2 h_i\) for \(1\le i \le m\) and \(g_j\) has odd order \(h_j\) for \(m+1 \le j \le k\). Note that \(h_i\) are odd integers with \(\lcm(h_1,\ldots,h_k)=h\). By what we already show in the previous point
		      \[
			      \ord(G_i) =
			      \begin{cases}
				      h_i  & 1 \le i \le m    \\
				      2h_i & m+1 \le i \le k
			      \end{cases}
		      \]
		      Then, by \autoref{th:3.9},
		      \[
			      \ord(F) = E = \lcm(h_1,\ldots,h_m,2h_{m+1}, \ldots 2h_k).
		      \]
		      Hence \(E=h=e/2\) if \(m=k\) and \(E=2h=e\) if \(m<k\).\qedhere
	\end{itemize}
\end{proof}

\begin{teor}{Characterization of a primitive polynomial by its order}{3.16}
	Let \(f \in \F_q[x]\) be a monic polynomial of degree \(m\) with \(f(0)\neq 0\). Then \(f\) is primitive over \(\F_q\) if and only if \(f\) has order \(q^m-1\).
\end{teor}

\begin{proof}
	\graffito{\("\Rightarrow"\)}If \(f\) is primitive then it is irreducible over \(F_q\) and, by \autoref{th:3.3}, its order is the order of one of its roots \(\a\) over \(\F_{q^m}\), which is \(q^m-1\) as \(\a\) is a primitive element of \(\F_{q^m}\) over \(\F_q\).
	
	\graffito{\("\Leftarrow"\)}Suppose \(\ord(f)=q^m-1\) and suppose, by contradiction, that \(f\) is reducible over \(\F_q\). Then either \(f=g^b\), with \(g\in\F_q[x]\) irreducible, or \(f=f_1 f_2\) with \(\GCD(f_1,f_2)=1\).
	\begin{itemize}
		\item Suppose \(f=g^b\), then \(\ord(f)=p^t\ord(g)\), then \(p \mid \ord(f)\), which is impossible as \(p \nmid q^m-1\).
		\item Suppose \(f=f_1 f_2\). \(f_1\) and \(f_2\) are monic polynomials in \(\F_q[x]\) with degree \(m_1,m_2\) and order \(e_1,e_2\), respectively. In particular
		      \[
			      e_1 \le q^{m_1}-1 \qquad\text{and}\qquad e_2 \le q^{m_2}-1.
		      \]
		      Therefore
		      \begin{align*}
			      (q^m-1) & = \ord(f) \le (q^{m_1}-1)(q^{m_2}-1) = q^{m_1+m_2}-1 - (q^{m_1}+q^{m_2}) \\
			              & = q^m-1 - (q^{m_1}+q^{m_2}) < q^m-1,
		      \end{align*}
		      which is impossible.\qedhere
	\end{itemize}
\end{proof}

\begin{lem}
	Let \(f\in\F_q[x]\) be a polynomial of degree \(m\) with \(f(0)\neq 0\).
	Let \(r\) be the least positive integer such that \(x^r \equiv a\) modulo \(f\), with \(a\in\F_q^*\). Then
	\[
		\ord(f) = h\,r,
	\]
	with \(h\) the order of \(a\) in \(\F_q^*\).
\end{lem}

\begin{proof}
	Let \(e=\ord(f)\). We have \(e\ge r\) as \(x^e \equiv 1\) modulo \(f\). If we perform the division with reminder between \(e\) and \(r\) we get
	\[
		e = s\,r+t \qquad\text{with }0 \le t < r.
	\]
	Therefore
	\[
		1 \equiv x^e \equiv x^{s\,r+t} \equiv (x^r)^s x^t \equiv a^s x^t \pmod{f}.
	\]
	Hence \(x^t \equiv 1/a^s\) modulo \(f\), where \(1/a^s \in \F_q\). But \(t<r\) contradicts the minimality of \(r\) unless \(t=0\). Therefore \(e = s\,r\). Moreover \(a^s \equiv 1\) and \(s\) is the order of \(a\) in \(\F_q^*\).
\end{proof}

\begin{teor}{}{3.18}
	Let \(f\in\F_q[x]\) be a monic polynomial of degree \(m\ge 1\) with \(f(0)\neq 0\). Then \(f\) is primitive over \(\F_q\) if and only if
	\begin{equation*}\tag{\(*\)}
		\begin{cases}
			(-1)^m f(0) \text{ is a primitive element of \(\F_q\)} \\
			x^{\frac{q^m-1}{q-1}}\equiv a \pmod{f}\text{ with \(a\in\F_q\)}
		\end{cases}
	\end{equation*}
	where \((q^m-1)/(q-1)\) is the least positive integer such that \(x^r \equiv a\) modulo \(f\).
	Moreover, if \(f\) is primitive over \(\F_q\), we have
	\[
		x^r \equiv (-1)^m f(0) \pmod{f}.
	\]
\end{teor}

\begin{proof}
	\graffito{\("\Rightarrow"\)}Suppose \(f\) primitive, consider \(\a\in V(f)\) which is a primitive element of \(\F_{q^m}\), therefore \(\ord(\a)=q^m-1\). Now if we compute the norm of \(\a\) we get
	\[
		\Nr_{\F_{q^m}/\F_q}(\a) = (-1)^m f(0) = \a^{\frac{q^m-1}{q-1}}.
	\]
	Then \((-1)^m f(0)\) is an element of \(\F_q\) with order \(q-1\), hence it is a primitive element of \(\F_q\).
	Since \(f\) is the minimal polynomial of \(\a\) and \(\a\) is a root of \(x^{(q^m-1)/(q-1)}-(-1)^m f(0)\), we get
	\[
		f \mid x^{\frac{q^m-1}{q-1}}-(-1)^m f(0) \iff x^{\frac{q^m-1}{q-1}} \equiv (-1)^m f(0) \pmod{f},
	\]
	then \(r \le (q^m-1)/(q-1)\). We know that \(\ord(f)=q^m-1\) and, by previous lemma, that \(\ord(f)\) is equal to \(\ord(a) r\), where \(a \in \F_q\). Therefore
	\[
		q^m-1 = \ord(f) = \ord(a) r \le (q-1) r \implies r = \frac{q^m-1}{q-1}.
	\]
	\graffito{\("\Leftarrow"\)}Suppose \((*)\) holds. DA FINIRE!!
\end{proof}
%%%%%%%%%%%%%%%%%%%%%%%%%%%%%%%%%%%%%%%%%%
%
%LEZIONE 20/03/2018 - QUINTA SETTIMANA (1)
%
%%%%%%%%%%%%%%%%%%%%%%%%%%%%%%%%%%%%%%%%%%
\section{Irreducible polynomials}

\begin{teor}{Factorization of \(x^{q^m}-x\)}{3.20}%
	Consider \(x^{q^m}-x \in \F_q[x]\) and let \(f \in \F_q[x]\) be a generic monic irreducible polynomial of degree \(d\), with \(d \mid m\). Then
	\[
		x^{q^m}-x = \prod f.
	\]
\end{teor}

\begin{proof}
	By \autoref{2.13}, we know that
	\[
		f \mid x^{q^m}-x \iff d \mid m.
	\]
	Moreover \((x^{q^m}-x)' = q^m x^{q^m-1}-1 = -1\), therefore
	\[
		\GCD\big(x^{q^m}-x, (x^{q^m}-x)'\big) = 1
	\]
	and \(x^{q^m}-x\) has only simple roots. Hence
	\[
		x^{q^m}-x = \prod f,
	\]
	where \(f\) are monic irreducible polynomials of degree \(d \mid m\).
\end{proof}

\begin{notz}
	Consider the set of monic irreducible polynomials of degree \(d\) in \(\F_q[x]\), we define
	\[
		N_q(d) = \#\Set{f \in \F_q[x] | f \text{ monic, irreducible, }\pd f = d}.
	\]
\end{notz}

\begin{cor}\label{3.21}
	Consider \(N_q(d)\) the number of monic irreducible polynomial of degree \(d\) in \(\F_q[x]\). Then
	\[
		q^m = \sum_{d\mid m} d\,N_q(d).
	\]
\end{cor}

\begin{defn}{M\"obius function}{3.22}\index{M\"obius function}
	The M\"obius function \(\m\) is an arithmetic function defined as
	\[
		\m(n) =
		\begin{cases}
			1      & n = 1                                                     \\
			(-1)^k & n = p_1 \cdot\ldots\cdot p_k, p_i \neq p_j \text{ primes} \\
			0      & p^2 \mid n, p \text{ prime}
		\end{cases}
	\]
\end{defn}

\begin{lem}
	The Dirichlet transformation of \(\m\) is given by
	\[
		\sum_{d\mid n} \m(d) =
		\begin{cases}
			1 & n = 1  \\
			0 & n > 1
		\end{cases}
	\]
\end{lem}

\begin{proof}
	Suppose \(n>1\), then
	\[
		\sum_{d \mid n} \m(d) = \sum_{\substack{d \mid n\\p^2 \mid d}} \m(d) + \sum_{\substack{d \mid n\\p^2 \nmid d,\,\fa p}}\m(d) = \sum_{\substack{d \mid n\\p^2 \nmid d,\,\fa p}}\m(d).
	\]
	Consider \(p_1,\ldots,p_k\) primes such that \(p_i \mid n\), then
	\begin{align*}
		\sum_{\substack{d \mid n                                                                        \\p^2 \nmid d,\,\fa p}}\m(d) & = \m(1) + \sum_{\substack{d \mid n\\d = p_i}}\m(d) + \sum_{\substack{d \mid n\\d = p_i p_j}}\m(d) + \ldots + \sum_{\substack{d \mid n\\d = p_1 \cdot\ldots\cdot p_k}}\m(d)\\
		 & = 1 + \binom{k}{1}(-1) + \binom{k}{2}(-1)^2 + \ldots \binom{k}{k}(-1)^k = \big(1+(-1)\big)^k \\
		 & = 0^k = 0.\qedhere
	\end{align*}
\end{proof}

\begin{teor}{M\"obius inversion formula}{3.24}
	Let \(h\) and \(H\) be two function from \(\N\) to an additive abelian group \(G\). Then
	\[
		H(n) = \sum_{d \mid n}h(d) \iff h(n) = \sum_{d\mid n}\m(d) H\left(\frac{n}{d}\right) = \sum_{d\mid n}\m\left(\frac{n}{d}\right)H(d).
	\]
\end{teor}

\begin{proof}
	\graffito{\("\Rightarrow"\)}We have
	\begin{align*}
		\sum_{d\mid n}\m(d) H\left(\frac{n}{d}\right) & = \sum_{d \mid n}\m(d) \sum_{\d\mid \frac{n}{d}}h(\d) = \sum_{\substack{d,\d \\n=d\,\d\,m\\m \ge 1}}\m(d)h(\d)\\
		                                              & = \sum_{\d \mid n}h(\d)\sum_{d \mid \frac{n}{\d}}\m(d),
	\end{align*}
	where, by previous lemma,
	\[
		\sum_{d \mid \frac{n}{\d}}\m(d) =
		\begin{cases}
			1 & \frac{n}{\d}=1 \iff \d=n \\
			0 & \frac{n}{\d}>1
		\end{cases}
	\]
	Hence, the last identity becomes
	\[
		\sum_{\d \mid n}h(\d)\sum_{d \mid \frac{n}{\d}}\m(d) = h(n) \cdot 1 = h(n).
	\]
	\graffito{\("\Leftarrow"\)}Similar to the other direction.
\end{proof}

\begin{oss}
	If \(G\) is a multiplicative group, the thesis becomes
	\[
		H(n) = \prod_{d \mid n}h(d) \iff h(n) = \prod_{d \mid n}H\left(\frac{n}{d}\right)^{\m(d)} = \prod_{d\mid n}H(d)^{\m(n/d)}.
	\]
	The proof is identical.
\end{oss}

\begin{teor}{Number of monic irreducible polynomial of given degree}{3.25}
	The number \(N_q(n)\) of monic irreducible polynomial of degree \(n\) in \(\F_q[x]\) is given by
	\[
		N_q(n) = \frac{1}{n}\sum_{d\mid n}\m(d)q^{n/d}.
	\]
\end{teor}

\begin{proof}
	Consider \(h,H\colon \Z \longrightarrow \Z\) with
	\[
		h(n) = n\,N_q(n) \qquad\text{and}\qquad H(n) = q^n.
	\]
	By \autoref{3.21} we know that
	\[
		q^n = \sum_{d\mid n} d\,N_q(d) \iff H(n) = \sum_{d\mid n} h(d).
	\]
	Then, using the inversion formula we get
	\[
		h(n) = \sum_{d\mid n}\m(d)H\left(\frac{n}{d}\right) \iff n N_q(n) = \sum_{d\mid n}\m(d)q^{n/d},
	\]
	from which the thesis.
\end{proof}

\begin{teor}{Factors of nth cyclotomic polynomial}{3.27}
	Let \(Q_n\in\F_q[x]\) be the nth cyclotomic polynomial, with \(p\nmid n\). Then
	\[
		Q_n(x) = \prod_{d\mid n}(x^d-1)^{\m(n/d)}.
	\]
\end{teor}

\begin{proof}
	Consider \(h,H\colon \Z \longrightarrow \F_q(x)\) with
	\[
		h(n) = Q_n(x) \qquad\text{and}\qquad H(n) = x^n-1.
	\]
	By \autoref{th:2.45} we know that
	\[
		x^n-1 = \prod_{d \mid n} Q_d(x) \iff H(n) = \prod_{d \mid n}h(d).
	\]
	Then, using the inversion formula for the multiplicative case, we get
	\[
		h(n) = \prod_{d\mid n} H(d)^{\m(n/d)} \iff Q_n(x) = \prod_{d \mid n} (x^d-1)^{\m(n/d)}.\qedhere
	\]
\end{proof}

\begin{teor}{Product of monic irreducible polynomials of given degree}{3.29}
	Let \(I(q,n)\) be the product of all monic irreducible polynomial of degree \(n\) in \(\F_q[x]\). Then
	\[
		I(q,n) = \prod_{d\mid n}(x^{q^d}-x)^{\m(n/d)}.
	\]
\end{teor}

\begin{proof}
	From \autoref{th:3.20} we know
	\[
		x^{q^n}-x = \prod_{d\mid n}I(q,d).
	\]
	Then it is enough to apply the multiplicative case of the inversion formula to obtain the thesis.
\end{proof}

\begin{ese}
	We want to compute the product of all irreducible polynomials of degree \(2\) in \(\F_q[x]\). By previous theorem we have
	\begin{align*}
		I(q,2) & = (x^q-x)^{\m(2)}(x^{q^2}-x)^{\m(1)} = (x^q-x)^{-1} (x^{q^2}-x) = \frac{x^{q^2}-x}{x^q-x}                     \\
		       & = \frac{x^{q^2-1}-1}{x^{q-1}-1} = \frac{(x^{q-1}-1)(x^{q\,(q-1)}+x^{(q-1)(q-1)}+\ldots+x^{q-1}+1)}{x^{q-1}-1} \\
		       & = x^{q\,(q-1)}+x^{(q-1)(q-1)}+\ldots+x^{q-1}+1.
	\end{align*}
	For example, if \(q=2\), then
	\[
		I(2,2) = x^2+x+1,
	\]
	which is then the only irreducible polynomial of degree \(2\) in \(\F_2[x]\).
\end{ese}

\begin{teor}{}{3.31}
	Let \(I(q,n)\) be the product of all monic irreducible polynomial of degree \(n\) in \(\F_q[x]\). Then
	\[
		I(q,n) = \prod_m Q_m(x),
	\]
	for all \(m\) for which \(m\mid q^n-1\) and \(n\) is the order of \(q\) modulo \(m\).
\end{teor}

The following are the main result we can easily deduce from this sections:
Let \(\a\in\F_{q^m}\) and let \(g\) be the minimal polynomial of \(\a\) over \(\F_q\). Suppose \(g\) has degree \(d\), then

\begin{pr}
	\(g\) is irreducible over \(\F_q\) and \(d \mid m\).
\end{pr}

\begin{pr}
	Let \(f\in\F_q[x]\), then \(f(\a)=0\) if and only if \(g \mid f\).
\end{pr}

\begin{pr}
	Let \(f\in\F_q[x]\) be a monic irreducible polynomial with \(f(\a)=0\), then \(f=g\).
\end{pr}

\begin{pr}
	\(g\) divides \(x^{q^d}-x\) and \(x^{q^m}-x\).
\end{pr}

\begin{pr}
	\(V(g)=\{\a,\a^q,\ldots,\a^{q^{d-1}}\}\) and \(g\) is the minimal polynomial of all these elements over \(\F_q\).
\end{pr}

\begin{pr}
	If \(\a\neq 0\), then \(\ord(g)=\ord_{\F_{q^m}^*}(\a)\).
\end{pr}

\begin{pr}
	\(g\) is a primitive polynomial over \(\F_q\) if and only if \(\a\) is a primitive element in \(\F_{q^d}\) if and only if \(a\) has order \(q^d-1\) in \(\F_{q^m}^*\).
\end{pr}